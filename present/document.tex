\documentclass{IEEEtran}
\usepackage[utf8]{vietnam}
\usepackage[backend=biber, sorting=ynt,autolang=other]{biblatex}
\usepackage{hyperref}
\usepackage{graphicx}
\usepackage{subcaption}
% Redefine bibliography strings for Vietnamese
\DefineBibliographyStrings{english}{%
  in          = {trong},
  andothers   = {và những người khác},
  pages       = {trang},
  page        = {trang},
  chapter     = {tập},
}

\renewcommand\IEEEkeywordsname{Từ khóa}
\addbibresource{references.bib}

\begin{document}

\title{AngularGrad: A New Optimization Technique for Angular Convergence of Neural Networks \\
\vspace{1cm}
    Các tóm tắt chính
}

\author{
    \IEEEauthorblockN{Nguyễn Hồng Sơn\IEEEauthorrefmark{1}\IEEEauthorrefmark{2}\\}
    \IEEEauthorblockA{\IEEEauthorrefmark{1}Trường Đại học Công nghệ Thông tin}\\
    \IEEEauthorblockA{\IEEEauthorrefmark{2}Email: sonnh.17@grad.uit.edu.vn}
}

\maketitle

\begin{abstract}
    Mạng nơ ron tích chập (CNN) được huấn luyện
    sử dụng các trình tối ưu hóa dựa trên độ dốc giảm dần ngẫu nhiên (SGD). Gần đây, trình tối ưu hóa ước lượng thời điểm thích ứng (Adam) đã
    trở nên rất phổ biến do động lực thích ứng của nó,
    giải quyết vấn đề độ dốc sắp chết của SGD. Tuy nhiên, hiện có
    trình tối ưu hóa vẫn không thể khai thác độ cong tối ưu hóa
    thông tin một cách hiệu quả. Bài báo này đề xuất một AngularGrad mới
    trình tối ưu hóa xem xét hành vi của hướng/góc của
    gradient liên tiếp. Đây là nỗ lực đầu tiên trong văn học
    để khai thác thông tin góc gradient ngoài
    kích cỡ. AngularGrad được đề xuất tạo ra điểm số để
    kiểm soát kích thước bước dựa trên thông tin góc gradient
    của các lần lặp trước đó. Vì vậy, các bước tối ưu hóa trở nên
    mượt mà hơn vì kích thước bước chính xác hơn của quá khứ ngay lập tức
    độ dốc được ghi lại thông qua thông tin góc. Hai
    các biến thể của AngularGrad được phát triển dựa trên việc sử dụng
    Các hàm tang hoặc Cosine để tính toán góc gradient
    thông tin. Về mặt lý thuyết, AngularGrad cũng thể hiện sự hối tiếc tương tự
    bị ràng buộc như Adam cho mục đích hội tụ. Tuy nhiên, rộng rãi
    các thử nghiệm được thực hiện trên các tập dữ liệu chuẩn so với các phương pháp tiên tiến cho thấy hiệu suất vượt trội của AngularGrad.
\end{abstract}

\begin{IEEEkeywords}
    
\end{IEEEkeywords}

\section{Đặt vấn đề}


\printbibliography[heading=bibintoc, title = {Tài liệu tham khảo}]
\end{document}
